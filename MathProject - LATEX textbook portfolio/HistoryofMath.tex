\section{History of Mathematics}

Mathematics has a long and storied history. Every society on earth has made their own mathematics and contributed to the state of the art today. Volumes and volumes have been written that take us from the beginning mathematics to the modern day. Today however, we will focus very closely on the sizeable contributions of ancient greece to the field of mathematics\newline

Information in this summary is gleaned from~\cite{MathHistory}. Please note that this summary is drawn in large part from a paper originally written for the history of math course offered at Indiana University East.

\subsection{Overview of Ancient Greek Mathematics}

The contributions of Greek mathematicians
to the body of knowledge concerning mathematics are impossible to overstate. They founded and formalized the idea of proof, which is so central to mathematics today. They also made massive strides in the field of geometry, which
will be a large part of our focus today. Additionally they made major advances in number theory
and even came close to stumbling upon some of the ideas of calculus.
Because of this, it is hard to find any field of mathematics which isn’t in some way influenced by the Greeks\newline

Mathematics in Ancient Greece can be largely divided into two broad categories based upon time.\newline

This period of time spans from around 600 BCE to around 146 BCE.
There
are literally dozens of eminent
mathematicians and philosophers hailing from Greece during this
period of time, but in the interest of
brevity, we’ll be limiting our discussion today to six giants
among giants:
Thales, Pythagoras, Aristotle, Euclid, Archimedes and Eratosthenes.
We’ll be
taking our study of these men in chronological order so as to better understand the developments
and contributions of Greek math.

\subsubsection{Thales of Miletus}
Thales of Miletus is often touted as the first Greek mathematician and moreover, the first
true scientist of the west. As such, it would seem fitting to start our story here. Thales was born In
624 BCE and lived into his eighties, passing away in 546 BCE. Much of our knowledge of Thales
is anecdotal and through the writings of others. What is certain is that he had an advanced (for the
time) conception of geometry and applied it to great effect. One of the early accomplishments
attributed to Thales was the prediction of a solar eclipse in 585 BCE. Given that there was no
observable cycle for solar eclipses, this is quite an accomplishment, though there is no record of
how he came to predict it. Another tale of Thales brilliance comes in the account of how he was able to measure
the heights of the pyramids to a reasonable degree of accuracy. He did it by measuring the length
of the shadow the pyramid cast when the shadow he cast was precisely as tall as he was.
This achievement implies a body of knowledge that was peerless in its time. First and most importantly
it means that he understood the idea of similar triangles, meaning that he recognized that when
him and his shadow formed an equilateral triangle, so too must the pyramid and its shadow.
This leads us neatly into the nitty-gritty of his actual contributions to the field. There is some contention
on the issue, but it is debated that Thales should be credited with the following 5 theorems of
elementary geometry.

\begin{theorem}
A circle is bisected by its diameter
\end{theorem}

This theorem is basic and quite handy as it gives us definite dimensions on any line which bisects a circle.

\begin{theorem}
The base angles of an isosceles triangle are equal
\end{theorem}

This demonstrates a more advanced understanding of the qualities of triangles than was common at the time.

\begin{theorem}
The angles between two intersecting straight lines are equal
\end{theorem}

\begin{theorem}
Two triangles are congruent if you have two and angles and one side equal to each other
\end{theorem}

This again demonstrates an understanding of trigonometry that was advanced for the time and extremely useful.

\begin{theorem}
An angle in a semicircle is a right angles
\end{theorem}

This list of theorems is absolutely fundamental to geometry and forms part of many other works
and proof that proceed it.
The last of these is fittingly called Thales’ theorem. Of course, he also
had a few more fantastic ideas which were less rooted in fact and rigor than speculation. He believed that
earthquakes were the result of the earth being rocked by waves as it floated in a vast ocean, and further believed that water was the
irreducible
substance from which all things were formed. While
this was incorrect, it was also the first recorded attempt to describe the materiel world in terms of
a small, finite number of different elements.

\subsubsection{Pythagoras}
Pythagoras
was an incredible, but somewhat
odd mathematician who lived from 569 BCE
to 475 BCE, having some overlap with Thales.
Of course, when we think of Pythagoras, our minds
instantly jump to the theorem known by children the world over $$a^2 + b^2 = c^2$$

with a and b being
the length of the sides
of a right triangle, and c being the length of the hypotenuse. Unfortunately
for Pythagoras, the theorem attributed to him was discovered by the Babylonians a millennium
earlier; however, it is possible that he may have been the first to rigorously prove
it. \newline

His most important innovation was something that today we take almost completely for granted: the
idea of abstraction. Before Pythagoras, mathematics was set in the concrete; with numbers being
used only as a representation of actual quantities. In a sense, this is part of why so much of early
math is concerned
with geometry. In a sense, Pythagoras invented the field of pure math by way
of divorcing numbers from the limited domain of representation, and allowed them to be objects
in their own right. This concept of abstraction allowed the idea of abstract mathematical proof to flourish. His fascination with numbers as objects arguably founded number theory, often touted as the queen of mathematics.\newline

 His love of music led him to attempt to codify music with mathematical formulae, which allowed him to discover the ratios of the keys and scales. He also believed that the motion of the planets was inherently quantifiable and formulaic, and related it to his work with music in what he called “The music of the spheres”.\newline
 
He also created an order known, appropriately enough, as the Pythagoreans. This group was one part academic, another part, monastic order, and another part of something akin to a cult. The tenets of the order are a bit hazy, but we do know that they held an almost religious reverence to the mystical math that they worked at. They also may have believed in the inherent magical qualities of certain numbers and had a miscellany of interesting beliefs regarding ritual. One very fascinating feature of the group was a belief that reality was inherently quantifiable, which is certainly a progressive thought for the time. But now we must jump forward by more than a century and take a look at Aristotle.

\subsubsection{Aristotle}

Aristotle presents us with a problem, since he was known primarily as a philosopher rather
than a mathematician. As a matter of fact, most if not all of the great Greek mathematicians were
considered philosophers, although the term
carried a broader meaning than it does today.\newline

Aristotle’s contributions to math are more general than the other people on our list. His
contribution to math lies in logic, without which mathematics could not exist.
He codified term logic, which was extremely influential and useful in every avenue of science.\newline

 He is also credited
with codifying the 3-line syllogistic argument, which is an excellent foundational idea for proof
through relation. Now we move on to one of the most prolific luminaries on our list
today: Euclid.

\subsubsection{Euclid}

Euclid’s life is an enigma; almost all we know of him can be summed up in this sentence:
He was born sometime in the
mid-4th
century BCE, taught in Alexandria, and died somewhere
towards the middle of the 3rd
century BCE.\newline

Euclid made sizeable and fantastic contributions to the field of
geometry. For Euclid, the study begins and ends with his book \textit{Elements}. Euclid’s \textit{Elements} was a
series of 13 books written on the subject of math and geometry, which is widely considered to be
the most influential textbook in the history of the written word.
The book works from a series of basic axioms, definitions, and postulates
and derives from there.\newline

It
touches upon geometric algebra, proportion, spatial geometry, and even
number theory; acting much like an ark, ferrying the sum of mathematical knowledge at the time
to us in the contemporary world. The importance of this work is impossible to overstate,
and while
we could continue to trace its influence upon the mathematics, we must now move on to the
eminently practical Archimedes.

\subsubsection{Archimedes}
Archimedes is one of the most spectacular mathematicians of history. Archimedes spent
much of his life wearing the hat of engineer rather than mathematician, but mathematics was where
his heart truly resided. His achievements span geometry and he even touched upon the ideas of
calculus, which would be codified over 1600 years after his death in 212 BCE. He used the method of exhaustion and even an early form of limits to prove many geometric theorems and relationships and found results including an approximation of pi which
are almost unreal in their accuracy.
Many of his achievement are more physical in nature.\newline

 He
invented the Archimedes screw,
which is a device utilizing a spiral (which was a near obsession
for Archimedes) to efficiently lift water. He also codified the Archimedes
principle which states
that the buoyant force exerted on an object in fluid is equal to the weight of the fluid displaced.
There is an apocryphal story
about Archimedes which relates to water and displacement.\newline

In short,
a crown was made for a temple
with the instructions to use pure gold in its
construction, but it was
suspected that the smiths might have alloyed the gold with a lesser metal to save on production
costs. Archimedes was asked to figure out whether such subterfuge had taken place without
damaging the crown.\newline

 Had it been a cube or some regular shape, it would have been easy as he
could simply use volume and weight to determine density. Archimedes was stumped and went to
take a bath. He observed that the water he displaced was equal to his volume.\newline

 He realized that he
could use this to determine the density of the crown and in his excitement he took to the streets
stark naked, shouting “Eureka!” meaning “I have found it”.
Unfortunately, Archimedes of Syracuse was killed when the Romans sacked his city.\newline

The story goes that he was drawing circles
in the dirt, working out some problem. A Roman soldier grew impatient and struck him down with
his last word being something to the effect of don’t disturb my circles. 

\subsubsection{Eratosthenes}

Last but not least we come to Eratosthenes of Cyrene.
Eratosthenes
main accomplishments
come in the disparate fields of geography and number theory. His best known accomplishments
involve his derivation of the circumference of the earth and the earth’s axial tilt both of which
he determined
to a high degree of accuracy.
His less widely known but also extremely important
accomplishment was the invention of the so called “Sieve of Eratosthenes” which is an extremely
efficient algorithm for finding prime numbers still in use today. The sieve works in an incredibly
simple manner. The algorithm works quite simply as follows:
\begin{theorem} Sieve of Eratosthenes\newline

First make a list of every number from 2 to some arbitrary end point n.\newline
Then cross out every second number (4,6,8, etc) after 2.\newline
Then cross out every third number from three, and every fifth from five.\newline
Repeat until you hit $\frac{n}{2}.$\newline
The numbers remaining are all the prime numbers in that span.
\end{theorem}
This allows you to find arbitrarily large prime numbers with little computation.\newline

However, it grows less efficient when the end point gets extremely large since it means you have to run through your number line many times. This simple algorithm is a great
accomplishment which at the time may have been
esoteric.\newline

 After all, a theorem regarding prime numbers does not seem to have an immediate application. However, today prime numbers form the basis for cryptography and computer security. It goes
to show that even the most exotic of results may eventually find an application.
