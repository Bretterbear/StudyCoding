
\section{Differential Equations} %Each summary that you do in this class will be a section.  Use \subsection and \subsubsection commands to create sections within the summary.

Differential Equations are a tremendously powerful tool in the mathematicians arsenal. In particular, the utility of differential equations in modelling systems is almost boundless.
The information for the following summary of Differential Equations is taken from~\cite{DiffEq}. Please note that the examples presented in the following summary section are also provided in the text as I no longer have notes from a Differential Equations course.

\subsection{Ordinary Differential Equations}

To put it simply, a differential equation is one in which derivatives are used as variables in an equation. In the simplest case, we have what is called an \emph{ordinary differential equation}.

\begin{definition}
An \emph{ordinary differential equation} is an equation relating an unknown function of one variable to one or more functions of its derivatives.
\end{definition}

Here are some examples of ordinary differential equations with $x$ as our unknown as a function of $t$. That is to say $x = x(t).$

\begin{example}
$$\frac{dx}{dt}=t^7\cos x$$
$$\frac{d^2 x}{dt^2} = x \frac{dx}{dt}$$
$$\frac{d^4 x}{dt^4} = -5 x^5$$
\end{example}

Each differential equation in this example is of a different \emph{order}.

\begin{definition}
The \emph{order} of a differential equation is the order of the highest derivative of the unknown function (dependent variable) that appears in the equation.
\end{definition}

In our example set of differential equations, the orders are first, second, and fourth order respectively.\newline

Very often, our dependent variable is a function of time, denoted as $t.$ Applications for this type of equation include

\begin{enumerate}
\item Population Dynamics
\item Mixture and Flow Problems
\item Electronic Circuits
\item Mechanical Vibrations and Systems
\end{enumerate}

We will now show an example problem regarding population dynamics, which should give us a good idea of the type of problems differential equations hopes to answer.

\begin{example} Population Dynamics Over Time\newline

Assume that the population of Washington, DC, grows due to births and deaths at the rate of $2\%$ per year and there is a net migration into the city of $15000$ people per year. Write a mathematical equation that describes the situation.\newline

SOLUTION. We let
\begin{center}
$x(t)=$ population as a function of time $t.$
\end{center}
From Calculus,
\begin{center}
$\frac{dx}{dt} =$ rate of change of the population $x$
\end{center}
In our example, $2\%$ growth means $2\%$ of the population $x(t).$ Thus, the population of Washington, DC satisfies the following $$\frac{dx}{dt} = 0.02x + 15000.$$
\end{example}

Physical laws often lead to differential equations which are quite handy. In our next example, we will look at the applications of differential equations to free-falling bodies in open air.

\begin{example} The Joys of Free-fall \newline

Newton's Law says $$F=ma$$ where m is constant mass, and a is acceleration, the second derivative of position. This allows us to rewrite Newton's Law as $$m \frac{d^2x}{dt^2} = F.$$ If the forces acting on a particular body going into free-fall are gravity (which we define as $-mg$ and the force of air resistance, which we know to be proportional to velocity ($\frac{dx}{dt}$, then our position at any given time satisfies the following second order differential equation
$$m\frac{d^2x}{dt^2} = -mg - c\frac{dx}{dt}$$. This is a bit underwhelming as a result. However, suppose we are tied to a spring which exerts an additional force proportional to our position that satisfies Hooke's Law. In this case, we will end up with the following, more interesting second order differential equation $$m\frac{d^2x}{dt^2} = -mg -kx -c\frac{dx}{dt}$$
\end{example}

Many of the physical laws concerning electricity give rise to differential equations as well. As such, we can model the behaviour of circuits with differential equations quite handily. For example, a simple RLC circuit can be modelled by the following second-order differential equation:
$$L\frac{d^2i}{dt^2}+R\frac{di}{dt}+\frac{1}{C}i = f(t)$$

But if we derive or are given a possible solution to a differential equation, how can we verify that what we have is actually a solution?

\begin{example} Showing that a Function is a Solution\newline
Verify that $x=3e^{t^2}$ is a solution of the following first-order differential equation
$$\frac{dx}{dt} = 2tx$$
We substitute $x=3e^{t^2}$ in both the left and right hand sides of our equation. this gives us $$\frac{d}{dt}(3e^{t^2}) = 6te^{t^2}.$$
Simplifying the left hand side, we end up with $$6te^{t^2}=6te^{t^2}.$$ This holds for all $t$, thus we can see that  $x=3e^{t^2}$  is a solution for our differential equation.

\end{example}

\subsubsection{Initial Value Problems}

The simplest first-order differential equation possible arises if the function $f(x,t)$ in $\frac{dx}{dt} = f(x,t)$ does not depend on the unknown solution giving us the differential equation $$\frac{dx}{dt} = f(t).$$ Luckily, the solution to such a system is quite apparent. We can find a solution by way of integration.\newline

However, we can see that integrating like this will give us an arbitrary constant. From this, we can see that differential equations can have an infinite number of possible solutions. A solution like this is called the \emph{general solution} of a differential equation because it is a sort of formula which gives all possible solutions.

 In the following example, we will solve a differential equation this way, and toss in the wrinkle of initial values to find a definite solution.

\begin{example} Introducing Initial Values\newline

Consider the following simple differential equation,
$$\frac{dx}{dt}	= t^2.$$
By integration, we obtain
$$x=\frac{1}{3}t^3 + c$$
where c is an arbitrary constant. Without further information, this would be where the problem would end.\newline

However, suppose that we are given the information $x=7$ at $t=2,$ that is to say, $x(2)=7.$ Applying this information to our general solution we get $$7 = \frac{8}{3} + c.$$ We can then determine that $c=\frac{13}{3}.$ Now we have a unique solution to our differential equation which satisfies our initial condition: $$x=\frac{1}{3}t^3+\frac{13}{3}.$$
\end{example}

If our equation is packaged to allow this kind of simple integration, then everything is good. However, suppose that we have a situation where using an indefinite integral is not appropriate. In this case, we can use a definite integral. If both sides of the differential equation $\frac{dx}{dt}=f(t)$ are integrated with respect to $t$ from $t_0$ to $t,$ this gives us $$\int\limits_0^1 \frac{dx}{d\bar{t}}d\bar{t} = \int\limits_0^1 f(\bar{t})d\bar{t}.$$ Note that we introduced a dummy variable and are no longer in t. By working out both sides and cancelling, this gives us $$x(t) = x(t_0) + \int\limits_{t_0}^t f(t)dt.$$ Now lets look at an example of this operation in action.

\begin{example} Using Indefinite Integrals\newline

Solve the differential equation 
$$\frac{dx}{dt} = e^{-t^2}.$$
Subject to the initial condition $x(3)=7.$\newline

The function $e^{-t^2}$ does note have an explicit antiderivative. Thus, we will solve using the definite integration from $t_0=3$ to $t=7.$ Using our previously found solution, we find that the solution to this initial value problem is $$x=7+\int\limits_3^t e^{-t^2}d\bar{t}.$$
\end{example}

We will end our discussion of differential equations with a very practical example from the area of mechanics.

\begin{example} Brake or Break\newline

There is a car going 76 meters per second down a road when the driver sees a deer down the road. The brakes are applied hard at $t=2.$ Due to inefficiencies in the braking system, acceleration is known to conform to $a=-12t^2$. How far does the car travel after the breaks are applied?\newline

Our differential equation is $$\frac{d^2x}{dt^2}=-12t^2$$
By integrating like we did in an earlier example we obtain
$$\frac{dx}{dt}=-4t^3 + c_1$$
Integrating again
$$x=-t^4 + c_1 t + c_2$$
Now we go back and apply our initial conditions to find a definite solution. We know that $x=0$ and $\frac{dx}{dt}=76$ at time $t=2.$\newline

By applying our velocity constraint we get $$76 = -4t^3 +c_1$$
showing us that $c_1 = 108.$ After this, we apply our position constraint in order to see $$0=-2^4 + 108t+c_2.$$ From this we get that $c_2 = -200.$\newline

We now solve the following equation $$\frac{dx}{dt} =-4t^3 +108 = 0$$ to find that our stopping time is $t=3$ Substituting $t=3$ this back into our equation $x=-t^4 + 108t -200$ giving us a stopping distance of 43 meters.
\end{example}


